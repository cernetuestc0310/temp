%%
%% Dibbler - a portable DHCPv6
%%
%% authors: Tomasz Mrugalski <thomson@klub.com.pl>
%%          Marek Senderski <msend@o2.pl>
%%
%% released under GNU GPL v2 only licence
%%
%% $Id: dibbler-devel-arch.tex,v 1.4 2008-08-29 00:07:38 thomson Exp $
%%

\section{Architecture}

General architecture is common between server, client and (to some extent)
relay. In all cases, classes are divided into several major groups:
\begin{description}
\item[IfaceMgr] -- Interface Manager. It represents all network interfaces present in the
  system. They're represented by TIfaceIface objects and stored in
  IfaceLst. Each interface has list of open sockets, represented with
  TIfaceSocket objects. There are also a number of auxiliary functions
  for getting proper interface. IfaceIface objects  also provide
  methods to add, update and remove addresses.
\item[AddrMgr] -- Address Manager. It is an address database, which
  stores all informations about clients, IAs and associated addresses.
\item[CfgMgr] -- Config Manager. It is being used to read
  configuration information from config file and provide those
  informations while runtime. Common mechanisms shared between server
  and client are scarce, so this base class is almost empty.
\item[TransMgr] -- Transmission Manager, sometimes called Transaction
  Manager. It is responsible for network interaction and core DHCPv6
  logic. It sends various messages when such need arise, matches received
  responses with sent messages, retransmits messages etc. It contains
  list of messages currently being trasmitted. 
\item[Messages] -- There is one parent class of all messages. It
  contains several basic functionalites common to all messages.
\item[Options] -- There are multiple option classes. Note that some
  classes are designed to represent one specific option
  (e.g. OptIAAddress) and other are not (e.g. OptAddrLst can contain
  address list, so it can be used as DNS Resolvers, SIP servers o NIS
  servers option). 
\item[Misc] -- This cathegory (or rather directory) contains various
  miscellanous classes and functions.
\end{description}

None of those classes is used directly. Client, server and
relay uses derived classes.

They are all created within DHCPClient or DHCPServer objects in client
or server, respectively. DHCPRelay object will perform similar
function for relays.

\subsection{Client Architecture}

Client is represented by a DHCPClient object. It contains 4 large
managers, each with its own functions. Also messages and options are
defined:

\begin{description}
\item[TClntIfaceMgr] -- contains client version of the IfaceMgr. Major
  difference is a TClntIfaceIface class, an enhanced version of the
  IfaceIface. It provides methods to set up various options on the
  physical interface. Those methods are used by Options representing
  options.
\item[TClntAddrMgr] -- Client version supports additional, client
  related functions, e.g. tentative timeout used in DAD procedure. It
  also simplifies database handling as there will always be only one
  client in the database.
\item[TClntCfgMgr] -- Client related parser. TClntCfgMgr and related
  objects are designed to provide easy access to parameters specified
  in the configuration file. ClntCfgIface is a very important class as
  most of the parameters is interface-specific.
\item[TClntTransMgr] -- Core logic of the Client. It uses all other
  managers to decide what actions should be taken at occuring
  circumstances, e.g. send REQUEST when there are less addresses
  assigned than specified in the configuration file.
\item[TClntMsg] -- All messages have client specific
  classes. Those objects are created as new messages are being
  sent. After server message reception, object is also created and
  passed to the original message. For example, client sends
  \msg{SOLICIT} message and server send \msg{ADVERTISE} message. Reply
  will be passed by invoking \verb+answer(msgAdvertise)+ method on the
  \verb+msgSolicit+ object.
\item[TClntOpt] -- There are client specific options defined. Each
  of those options has \verb+doDuties()+ method which is called if this
  option was received in a proper reply message from the server. It
  calls appropriate methods in TClntIfagrMgr which set specific options
  in the system.
\end{description}

\subsection{Server Architecture}

Server is represented by a DHCPServer object. It contains 4 large
managers, each with its own functions. Also SrvMessages and SrvOptions
are defined:
\begin{description}
\item[TSrvIfaceMgr] -- contains server version of the IfaceMgr. There
  are almost no modificiation compared to common version.

\item[TSrvAddrMgr] -- Client version supports additional, client
  related functions, e.g. tentative timeout used in DAD procedure. It
  also simplifies database handling as there will always be only one
  client in the database.
\item[TSrvCfgMgr] -- Client related parser. TSrvCfgMgr and related
  objects are designed to provide easy access to parameters specified
  in the configuration file. SrvCfgIface is a very important class as
  most of the parameters is interface-specific.
\item[TSrvTransMgr] -- Core logic of the client. It uses all other
  managers to decide what actions should be taken at occuring
  circumstances, e.g. send REQUEST when there are less addresses
  assigned than specified in the configuration file.
\item[TSrvMsg] -- Server version of the messages. Each time server
  receives a message, TSrvMsg is created. Depending of its type,
  TSrvAdvertise of TSrvReply message is created. As parameter to its
  contructor original message is passed. After creating message, it is
  sent back to the client and stored for possible retransmission
  purposes.
\item[TSrvOpt] -- Server version of the Option representing
  objects. They are just used to store data, so they are considerably
  simpler than client versions.
\end{description}

\subsection{Relay Architecture}
Preliminary relay version was available in the 0.4.0 release. It
consists of serveral simple blocks:
\begin{description}
\item[TRelIfaceMgr] -- contains relayr version of the IfaceMgr. There
  are almost no modificiation compared to common version, execept
  decodeMsg() and decodeRelayRepl() methods.
\item[TRelCfgMgr] -- Relay related parser. TRelCfgMgr and related
  objects are designed to provide easy access to parameters specified
  in the configuration file. RelCfgIface is a very important class as
  most of the parameters is interface-specific.
\item[TRelTransMgr] -- It's plain simple manager. It's only function
  is to relay received message on all interfaces.
\item[TRelMsg] -- From the relay's point of view, all messages fall to
  one of 3 categories: Generic (i.e. not encapsulated) messages,
  RelayForw (already forwarded by some other relay) and RelayRepl
  (replies from server). Most of the messages is threated as generic
  message.
\item[TRelOpt] -- Similar approach is used to handle options. Expect
  RELAY\_MSG option (which contains relayed message) and interface-id
  option (which contains identifier of the interface), all options are
  threated as generic options, which are handled transparently.
\end{description}
