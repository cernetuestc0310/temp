%%
%% $Id: dibbler-user-faq.tex,v 1.11 2007-09-06 23:10:40 thomson Exp $
%%
%% $Log: dibbler-user-faq.tex,v $
%% Revision 1.11  2007-09-06 23:10:40  thomson
%% Changes related to 0.6.1 release: small fix in SrvOptIA_NA,
%% added info about cross-compilation, new constants added.
%%
%% Revision 1.10  2006-10-04 22:03:35  thomson
%% *** empty log message ***
%%
%% Revision 1.9  2005-08-07 17:54:52  thomson
%% Minor changes related to 0.4.1 release.
%%
%% Revision 1.8  2005/03/15 23:02:31  thomson
%% 0.4.0 release.
%%
%% Revision 1.7  2005/02/01 00:57:36  thomson
%% no message
%%
%% Revision 1.6  2004/12/04 23:47:57  thomson
%% Work around for bug #56 described.
%%
%% Revision 1.5  2004/12/01 21:58:43  thomson
%% *** empty log message ***
%%
%% Revision 1.4  2004/11/28 11:14:07  thomson
%% RFCs and drafts added, clarification about option values
%%
%% Revision 1.3  2004/10/25 20:45:54  thomson
%% Option support, parsers rewritten. ClntIfaceMgr now handles options.
%%
%% Revision 1.2  2004/06/19 10:24:59  thomson
%% Hyperlinks in PDF, building process modified
%%


\section{Frequently Asked Questions}

Soon after initial Dibbler version was released, feedback from user
regarding various things started to appear. Some of the questions were
common enough to get into this section.

\subsection{Common Questions}

%% Q1 
\Q Why client does not configure routing after assigning addresses, so
I cannot e.g. ping other hosts?

\A It's rather difficult problem. DHCP's job is to obtain address and
it exactly does that. To ping any other host, routing should be 
configured. And this should be done using Router Advertisements. It's
kinda odd, but that's the way it was meant to work. If there will be
requests from users, I'll think about some enchancements.

Important note: This behaviour has changed in the 0.5.0 release. See
\emph{strict-rfc-no-routing} directive in the \ref{client-cfg-file}
section. 

%% Q2
\Q Dibbler server receives SOLICIT message, prints information about
ADVERTISE/REPLY transmission, but nothing is actually transmitted. Is
this a bug?

\A Are you sure that your client is behaving properly? Before any IPv6
packet (that includes DHCPv6 message) is transmitted, recipient
reachabity is checked (using Neighbor Discovery protocol 
\cite{rfc2461}). Server sends Neighbor Solicititation message and
waits for client's Neighbor Advertisement. If that is not transmitted,
even after 3 retries, server gives up and doesn't transmit IPv6 packet
(DHCPv6 reply, that is) at all. Being not able to respond to the
Neighbor Discovery packets may indicate invalid client behavior. 

%% Q3
\Q Dibbler sends some options which have values not recognized by the
Ethereal/Wireshark or by other implementations. What's wrong?

\A DHCPv6 is a relatively new protocol and additional options are in a
specification phase. It means that until standarisation process is
over, they do not have any officially assigned numbers. Once
standarization process is over (and RFC document is released), this
option gets an official number. 

There's pretty good chance that different implementors may choose
diffrent values for those not-yet officialy accepted options. To
change those values in Dibbler, you have to modify file
misc/DHCPConst.h and recompile server or client. See Developer's
Guide, section \emph{Option Values} for details.

Currently options with assigned values are: 
\begin{itemize}
\item RFC3315: \opt{CLIENT\_ID},
\opt{SERVER\_ID}, \opt{IA\_NA}, \opt{IAADDR}, \opt{OPTION\_REQUEST},
\opt{PREFERENCE}, \opt{ELAPSED}, \opt{STATUS\_CODE},
\opt{RAPID-COMMIT}, \opt{IA\_TA}, \opt{RELAY\_MSG}, \opt{AUTH\_MSG},
\opt{USER\_CLASS}, \opt{VENDOR\_CLASS}, \opt{VENDOR\_OPTS},
\opt{INTERFACE\_ID}, \opt{RECONF\_MSG}, \opt{RECONF\_ACCEPT};
\item RFC3319:
\opt{SIP\_SERVERS}, \opt{SIP\_DOMAINS};
\item RFC3646: \opt{DNS\_RESOLVERS},
\opt{DOMAIN\_LIST};
\item RFC3633: \opt{IA\_PD}, \opt{IA\_PREFIX};
\item RFC3898:
\opt{NIS\_SERVERS}, \opt{NIS+\_SERVERS}, \opt{NIS\_DOMAIN},
\opt{NIS+\_DOMAIN}. 

\end{itemize}
Take note that Dibbler does not support all of
them. There are several options which currently does not have values
assigned (in parenthesis are numbers used in Dibbler):
\opt{NTP\_SERVERS} (40), \opt{TIME\_ZONE} (41), \opt{LIFETIME} (42),
\opt{FQDN} (43).

%% Q4
\Q I can't get (insert your trouble causing feature here) to
work. What's wrong? 

\A Go to the project \url{http://klub.com.pl/dhcpv6/}{homepage} and
browse \href{http://klub.com.pl/lists/dibbler/}{list archives}. If
your problem was not reported before, please don't hesitate to write
to the
\href{http://klub.com.pl/cgi-bin/mailman/listinfo/dibbler}{mailing
  list} or \href{mailto:thomson(at)klub.com.pl}{contact author}
directly. 

\subsection{Linux specific questions}

%% Q1
\Q I can't run client and server on the same host. What's wrong?

\A First of all, running client and server on the same host is just
plain meaningless, except testing purposes only. There is a problem
with sockets binding. To work around this problem, consult Developer's
Guide, Tip section how to compile Dibbler with certain options.

%% Q2
\Q After enabling unicast communication, my client fails to send
REQUEST messages. What's wrong?

\A This is a problem with certain kernels. My limited test capabilites
allowed me to conclude that there's problem with 2.4.20
kernel. Everything works fine with 2.6.0 with USAGI patches. Patched 
kernels with enhanced IPv6 support can be downloaded from
\url{http://www.linux-ipv6.org/}. Please let me know if your kernel
works or not.

\subsection{Windows specific questions}

%% Q1
\Q After installing \emph{Advanced Networking Pack} or \emph{Windows XP
  ServicePack2} my DHCPv6 (or other IPv6 application) stopped
   working. Is Dibbler compatible with Windows XP SP2?

\A Both products (Advanced Networking Pack as well as Service Pack 2
for Windows XP) provide IPv6 firewall. It is configured by default to
reject all incoming IPv6 traffic. You have to disable this
firewall. To disable firewall on the ,,Local Area Connection''
interface, issue following command in a console:

\begin{verbatim}
netsh firewall set adapter "Local Area Connection" filter=disable
\end{verbatim}

%% Q2
\Q Server or client refuses to create DUID. What's wrong?

\A Make sure that you have at least one up and running interface with
at least 6 bytes long MAC address. Simple ethernet or WIFI card
matches those requirements. Note that network cable must be plugged
(or in case of wifi card -- associated with access point), otherwise
interface is marked as down.

%% Q3
\Q Is Microsoft Vista supported?

\A Unfortunately, Vista is not supported yet. There is a DHCPv6 client
already available in the Microsoft Vista. If time permits, next (0.6.2
probably) version will have support for Vista.


